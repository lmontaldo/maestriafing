\documentclass[12pt]{article}
\usepackage{amsmath} % for the equation* environment
% Set the font (output) encodings
\usepackage[T1]{fontenc}
% Spanish-specific commands
\usepackage[spanish]{babel}
\let\[\relax \let\]\relax % avoid warnings in the log file
\setlength{\parskip}{5pt}
\usepackage{setspace}
% Set custom line spacing (e.g., 1.2)
\setstretch{1.1}
%%
\usepackage{witharrows}
\usepackage{enumitem}
\usepackage{amsmath}
\usepackage{hyperref}
\usepackage{graphicx}
\usepackage[a4paper, margin=1in]{geometry}
\usepackage{placeins}
\usepackage{booktabs}
\usepackage{siunitx}
\usepackage{booktabs}
\usepackage{array} % for better array table cells
%%%%%%%%%%%%%%%%%%%%%%%%%%%%%%%%%%%%%%%%%%%%%%%%%%%%%%%%%%%%%%%%%%%
%%%%%%%%%%%%%%%%%%%%%%%%%%%%%%%%%%%%%%%%%%%%%%%%%%%%%%%%%%%
\begin{document}
\section{Análisis de los componentes del IPC}


Siguiendo la \href{https://www5.ine.gub.uy/documents/Estad%C3%ADsticasecon%C3%B3micas/PDF/IPC/IPC%20NOTA%20METODOLOGICA%20BASE%202010.pdf}{nota metodológica del IPC con base Diciembre de 2010}, para su cálculo, se registran mensualmente los precios de un conjunto de bienes y servicios seleccionados a partir de la estructura del gasto de consumo de los hogares de las regiones urbanas de Uruguay. Para esto, se considera el precio comprador, es decir, se toma el precio contado pagado efectivamente incluyendo los impuestos indirectos abonados por el comprador.  

En el año 2010, como innovación se adoptó la Clasificación del Consumo Individual por Finalidades (\textit{CCIF}) que clasifica el
gasto en cada uno de los productos y servicions siguiendo una estructura de división, grupo y clase\footnote{También, se incluyó un nivel inferior de agrupación  denominado familia que
responde a un criterio de afinidad entre los productos incluidos.}. En suma, la cantidad de productos relevados alcanza a 374 en todo el país. En total, se tienen doce divisiones 

\begin{table}[h!] % the "h!" tells LaTeX to place the table Here, right at this location
\centering
\caption{Divisiones del IPC}
\label{tab:div}
\begin{tabular}{r m{12cm}}
\toprule
código & Divisiones \\
\midrule
1 & alimentos y bebidas no alcohólicas \\
2 & bebidas alcoholicas, tabaco y estupefacientes \\
3 & prendas de vestir y calzado \\
4 & vivienda \\
5 & muebles, artículos para el hogar y para la conservación ordinaria del hogar \\
6 & salud \\
7 & transporte \\
8 & comunicaciones \\
9 & recreación y cultura \\
10 & educación \\
11 & restaurantes y hoteles \\
12 & bienes y servicios diversos \\
\bottomrule
\end{tabular}
\end{table}

Las divisiones se desagregan en grupos, clases, familias y productos. Considerando el universo de 374 artículos de la nueva canasta, $34,5\%$ se clasificaron como \textit{homogéneos}, $55,3\%$ como \textit{heterogéneos} y $10,2\%$ como \textit{especiales}, de los cuales el $63,2\%$ se considera como \textit{especiales homogéneos} y el $36.8\%$ como \textit{especiales heterogéneos}.

Los \textit{artículos homogéneos} son aquellos para los cuales es posible definir una especificación única que pueda ser encontrada en los diferentes establecimientos informantes de precios por períodos extensos de tiempo. Los \textit{artículos heterogéneos} implican definiciones genéricas o abiertas, que cubren productos no equivalentes a través de los diferentes puestos de venta. Es decir, no es posible hacer una descripción cerrada del bien, que pueda ser relevada en diferentes informantes y por un período
extenso de tiempo. Incluye especialmente a aquellos bienes o servicios afectados por cambios en
la moda, por cambios tecnológicos y aquellos que pueden tener una gama de precios con
diferencias significativas.
Se definen cómo \textit{artículos especiales} por el tratamiento que los mismos tienen respecto al
cálculo, ya que requieren un procedimiento particular. Asimismo den
tro de éstos artículos se
diferencian los artículos \textit{especiales homogéneos} y \textit{especiales heterogéneo}s. Fueron considerados
\textit{bienes especiales}: las tarifas públicas, los alquileres, los gastos comunes, los impuestos
municipales, la patente de automóviles, los seguros, el transporte departamental e
interdepartamental, el taxi, la telefonía celular, los juegos de azar, los diarios y revistas, la
televisión por cable, etc.
Los \textit{especiales homogéneos} son aquellos artículos especiales para los cuales es posible
calcular un precio medio de manera que el mismo tenga sentido. En general son productos cuyos
precios están fijados por uno o pocos oferentes. Por ejemplo: comprende algunas tarifas públicas,
servicios de transporte, etc.
Los \textit{especiales heterogéneos} son aquellos artículos especiales para los cuales resulta imposible
calcular un precio medio de manera que éste tenga sentido. Son productos que están sujetos a
continua innovación, son ofrecidos por un conjunto de oferentes o los distintos componentes del
mismo tienen tal heterogeneidad de manera que cualquier precio medio que pueda
confeccionarse para los mismos carece de sentido interpretativo.

En este estudio se van a tomar las clases como los componentes del IPC. En total,  son 88 componentes y la cantidad de los mismos varía entre divisiones. 
En un primer momento, se transforman las series a logaritmos para luego ser normalizadas. Posteriormente, se agruparon las clases según divisiones y se realiza un análisis gráfico preliminar de las mismas.

\clearpage
En el cuadro~\ref{tab:compo_01}, se tienen los siguientes componentes o clases para la división \textit{Alimentos y bebidas no alcohólicas}.

%%% 1 & Alimentos y bebidas no alcohólicas
\begin{table}[h!]
\caption{Alimentos y bebidas no alcohólicas}
\label{tab:compo_01}
\begin{tabular}{l m{10cm}}
\toprule
componentes & División 01 \\
\midrule
c0111 & Pan y cereales \\
c0112 & Carne \\
c0113 & Pescado \\
c0114 & Leche, queso y huevos \\
c0115 & Aceites y grasas \\
c0116 & Frutas \\
c0117 & Legumbres y Hortalizas \\
c0118 & Azúcar, mermelada, miel, chocolate y dulces de azúcar \\
c0119 & Productos alimenticios n.e.p. \\
c0121 & Café, te y cacao \\
c0122 & Aguas minerales, refrescos, jugos de frutas y de legumbres \\
\bottomrule
\end{tabular}
\end{table}


\begin{figure}[h!]
    \caption{Alimentos y bebidas no alcohólicas: 01}
    \label{fig:c01_raw}
    \centering
    \includegraphics[width=1\textwidth]{Metodologia/componentes_groups_jtos/c01_raw.png}
\end{figure}

La única serie que a simple vista parecería presentar evidentes cambios de pendiente
es $c0121$  relativa a \textit{Café, te y cacao}. A pesar de lo anterior, no se van a estudiar cambios estructurales de las series de forma individual por el momento.


%2 & bebidas alcoholicas, tabaco y estupefacientes \\


\clearpage
\begin{table}[h!]
\caption{Bebidas alcoholicas, tabaco y estupefacientes}
\label{tab:compo_02}
\begin{tabular}{l m{10cm}}
\toprule
componentes & División 02 \\
\midrule
c0211 & Bebidas destiladas \\
c0212 & Vino \\
c0213 & Cerveza \\
c0220 & Tabaco \\
\bottomrule
\end{tabular}
\end{table}

\begin{figure}[h!]
    \caption{Bebidas alcoholicas, tabaco y estupefacientes: 02}
    \label{fig:c02_raw}
    \centering
    \includegraphics[width=1\textwidth]{Metodologia/componentes_groups_jtos/c02_raw.png}
\end{figure}

Es de esperar que los valores escalonados en la serie $c0220$ vinculada a \textit{Tabaco} responda a las políticas relacionadas con la reducción de demanda del mismo, a su política tributaria y a las medidas que se toman al respecto en cada momento. 

\clearpage
%3 & prendas de vestir y calzado \\

\begin{table}[h!]
\caption{Prendas de vestir y calzado}
\label{tab:compo_03}
\begin{tabular}{l m{10cm}}
\toprule
componentes & División 03 \\
\midrule
c0312 & Prendas de vestir \\
c0314 & Limpieza, reparación y alquiler de prendas de vestir \\
c0321 & Zapatos y otros calzados \\
c0322 & Reparación y alquiler de calzado \\
\bottomrule
\end{tabular}
\end{table}

\begin{figure}[h!]
    \caption{Prendas de vestir y calzado: 03}
    \label{fig:c03_raw}
    \centering
    \includegraphics[width=1\textwidth]{Metodologia/componentes_groups_jtos/c03_raw.png}
\end{figure}
%4 & vivienda \\

\clearpage
\begin{table}[h!]
\caption{vivienda}
\label{tab:compo_04}
\begin{tabular}{l m{10cm}}
\toprule
componentes & División 04 \\
\midrule
c0411 & Alquileres efectivos pagados por los inquilinos \\
c0431 & Materiales para la conservación y la reparación de la vivienda \\
c0432 & Servicios para la conservación y la reparación de la vivienda \\
c0441 & Suministro de agua \\
c0442 & Recogida de basuras \\
c0443 & Alcantarillado \\
c0444 & Otros servicios relacionados con la vivienda n.e.p. \\
c0451 & Electricidad \\
c0452 & Gas \\
c0454 & Combustibles sólidos \\
\bottomrule
\end{tabular}
\end{table}

\begin{figure}[h!]
    \caption{vivienda: 04}
    \label{fig:c04_raw}
    \centering
    \includegraphics[width=1\textwidth]{Metodologia/componentes_groups_jtos/c04_raw.png}
\end{figure}

Los comportamientos de las series $c0441$ correspondiente a \textit{Suministro de agua} y $c0451$ a \textit{Electricidad}, respectivamente, responden a las políticas fijadas por sus respectivos entes reguladores.



\clearpage
%5 & muebles, artículos para el hogar y para la conservación ordinaria del hogar \\

\begin{table}[h!]
\caption{Muebles, artículos para el hogar y para la conservación ordinaria del hogar}
\label{tab:compo_05}
\begin{tabular}{l m{10cm}}
\toprule
componentes & División 05 \\
\midrule
c0511 & Muebles y accesorios \\
c0520 & Productos textiles para el hogar \\
c0531 & Artefactos para el hogar grandes, eléctricos o no \\
c0533 & Reparación de artefactos para el hogar \\
c0540 & Artículos de vidrio y cristal, vajilla y utensilios para el hogar \\
c0551 & Herramientas y equipo grandes \\
c0552 & Herramientas pequeñas y accesorios diversos \\
c0561 & Bienes para el hogar no duraderos \\
c0562 & Servicios domésticos y para el hogar\\
\bottomrule
\end{tabular}
\end{table}

\begin{figure}[h!]
    \caption{Muebles, artículos para el hogar y para la conservación del hogar: 05}
    \label{fig:c05_raw}
    \centering
    \includegraphics[width=1\textwidth]{Metodologia/componentes_groups_jtos/c05_raw.png}
\end{figure}


\clearpage
%6 & salud \\

\begin{table}[h!]
\caption{Salud}
\label{tab:compo_06}
\begin{tabular}{l m{10cm}}
\toprule
componentes & División 06 \\
\midrule
c0611 & Productos farmacéuticos \\
c0613 & Artefactos y equipo terapéuticos \\
c0621 & Servicios médicos \\
c0622 & Servicios dentales \\
c0623 & Servicios paramédicos \\
c0630 & Servicios de hospital \\
c0690 & Servicios médicos mutuales y colectivos \\
\bottomrule
\end{tabular}
\end{table}
\begin{figure}[h!]
    \caption{Salud: 06}
    \label{fig:c06_raw}
    \centering
    \includegraphics[width=1\textwidth]{Metodologia/componentes_groups_jtos/c06_raw.png}
\end{figure}

\clearpage
%7 & transporte \\
\begin{table}[h!]
\caption{Transporte}
\label{tab:compo_07}
\begin{tabular}{l m{10cm}}
\toprule
componentes & División 07 \\
\midrule
c0711 & Vehículos a motor \\
c0712 & Motocicletas \\
c0713 & Bicicletas \\
c0721 & Piezas de repuesto y accesorios para equipo de transporte personal \\
c0722 & Combustibles y lubricantes para equipo de transporte personal \\
c0723 & Conservación y reparación de equipo de transporte personal \\
c0724 & Otros servicios relativos al equipo de transporte personal \\
c0732 & Transporte de pasajeros por carretera \\
c0733 & Transporte de pasajeros por aire \\
c0734 & Transporte de pasajeros por mar y cursos de agua interiores \\
c0735 & Transporte combinado de pasajeros \\
c0736 & Otros servicios de transporte adquiridos \\
\bottomrule
\end{tabular}
\end{table}
\begin{figure}[h!]
    \caption{Transporte: 07}
    \label{fig:c07_raw}
    \centering
    \includegraphics[width=1\textwidth]{Metodologia/componentes_groups_jtos/c07_raw.png}
\end{figure}

\clearpage
%8 & comunicaciones \\
\begin{table}[h!]
\caption{Comunicaciones}
\label{tab:compo_08}
\begin{tabular}{l m{10cm}}
\toprule
componentes & División 08 \\
\midrule
c0810 & Servicios postales \\
c0820 & Equipo telefónico y de facsímile \\
c0830 & Servicios telefónicos y de facsímile \\
\bottomrule
\end{tabular}
\end{table}
\begin{figure}[h!]
    \caption{Comunicaciones: 08}
    \label{fig:c08_raw}
    \centering
    \includegraphics[width=1\textwidth]{Metodologia/componentes_groups_jtos/c08_raw.png}
\end{figure}

\clearpage
%9 & recreación y cultura \\
\begin{table}[h!]
\caption{recreación y cultura}
\label{tab:compo_09}
\begin{tabular}{l m{10cm}}
\toprule
componentes & División 09 \\
\midrule
c0911 & Equipo para la recepción, grabación y reproducción de sonidos e imágenes \\
c0912 & Equipo fotográfico, cinematográfico e instrumentos ópticos \\
c0913 & Equipo de procesamiento e información \\
c0914 & Medios para grabación \\
c0931 & Juegos, juguetes y aficiones \\
c0933 & Jardines, plantas y flores \\
c0934 & Animales domésticos y productos conexos \\
c0935 & Servicios de veterinaria y de otro tipo para animales domésticos \\
c0941 & Servicios de recreación y deportivos \\
c0942 & Servicios culturales \\
c0943 & Juegos de azar \\
c0951 & Libros \\
c0952 & Diarios y periódicos \\
c0954 & Papel y útiles de oficina y materiales de dibujo \\
c0960 & Paquetes turísticos \\
\bottomrule
\end{tabular}
\end{table}
\begin{figure}[h!]
    \caption{Recreación y culturas: 09}
    \label{fig:c09_raw}
    \centering
    \includegraphics[width=1\textwidth]{Metodologia/componentes_groups_jtos/c09_raw.png}
\end{figure}

La serie que presenta un comportamiento evidentemente particular es $c0943$ correspondiente a \textit{Juegos de azar}.


%%%10 & educación \\
\clearpage
\begin{table}[h!]
\caption{Educación}
\label{tab:compo_10}
\begin{tabular}{l m{10cm}}
\toprule
componentes & División 10 \\
\midrule
c1010 & Enseñanza preescolar o enseñanza primaria \\
c1020 & Enseñanza secundaria \\
c1040 & Enseñanza terciaria \\
c1050 & Enseñanza no atribuíble a ningún nivel \\
\bottomrule
\end{tabular}
\end{table}
\begin{figure}[h!]
    \caption{Educación: 10}
    \label{fig:c10_raw}
    \centering
    \includegraphics[width=1\textwidth]{Metodologia/componentes_groups_jtos/c10_raw.png}
\end{figure}
%%%11 & Restaurantes y hoteles \\
\clearpage
\begin{table}[h!]
\caption{Restaurantes y hoteles}
\label{tab:compo_11}
\begin{tabular}{l m{10cm}}
\toprule
componentes & División 11 \\
\midrule
c1111 & Restaurantes, cafés y establecimientos similares \\
c1112 & Comedores \\
c1120 & Servicios de alojamiento \\
\bottomrule
\end{tabular}
\end{table}
\begin{figure}[h!]
    \caption{Restaurantes y hoteles: 11}
    \label{fig:c11_raw}
    \centering
    \includegraphics[width=1\textwidth]{Metodologia/componentes_groups_jtos/c11_raw.png}
\end{figure}
%%%12 & Bienes y servicios diversos \\
\clearpage
\begin{table}[h!]
\caption{Bienes y servicios diversos}
\label{tab:compo_12}
\begin{tabular}{l m{10cm}}
\toprule
componentes & División 12 \\
\midrule
c1211 & Salones de peluquería y establecimientos de cuidados personales \\
c1213 & Otros aparatos, articulos y productos para la atención personal \\
c1232 & Otros efectos personales \\
c1252 & Seguro relacionado con la vivienda \\
c1254 & Seguro relacionado con el transporte \\
c1270 & Otros servicios n.e.p. \\
\bottomrule
\end{tabular}
\end{table}
\begin{figure}[h!]
    \caption{Bienes y servicios diversos: 12}
    \label{fig:c12_raw}
    \centering
    \includegraphics[width=1\textwidth]{Metodologia/componentes_groups_jtos/c12_raw.png}
\end{figure}


%%%%%%%%%%%%%%%%%%%%%%%%%%%%%%%%%
\end{document}
