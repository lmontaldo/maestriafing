\chapter{Introduction}


En este estudio se busca analizar en qué medida las perturbaciones en los precios internacionales de ciertos bienes transables denominados \textit{commodities} afectan a la inflación uruguaya a través de sus distintos componentes. 

La economía uruguaya presenta ciertas características particulares. Es una economía es pequeña y abierta, tomadora de precios en los mercados internacionales y exportadora de bienes primarios  Según fuentes oficiales \footnote{\href{https://fichapais.uruguayxxi.gub.uy/}{Ficha País, Uruguay XXI}}, en el año 2020, mientras que los tres principales bienes de exportanción se concentraban alrededor de productos cárnicos, celulosa y habas, los tres principales productos de importación en orden descendente fueron los combustibles, maquina y equipos de transporte. Por estos motivos,  sería interesante evaluar el grado de vulnerabilidad de la misma frente a los cambios en los precios de algunos de los principales bienes primarios transables a nivel global.


El presente estudio se enmarca en la literatura relacionada a los modelos dinámicos de factores. En un primer momento, se va a emplear un modelo de vectores autorregresivos aumentado por factores, conocido como FAVAR\footnote{ Factor-Augmented VAR model}. Éstos modelos son una extensión de los modelos VAR \footnote{ Vector autoregression model} 
y son útiles cuando se cuenta con un gran número de variables, porque permiten hacer frente al problema de la \textit{maldición de dimensionalidad} presente en los modelos VAR. Posteriormente se van a comparar los resultados de éste enfoque clásico con los de un modelo más reciente de factores dinámicos profundo ($D^2FM$)\footnote{Deep Dynamic Factor Model}. Este tipo de modelos se ubica en el marco de redes neuronales profundas permitiendo no linearidades entre los factores y las variables observables. Se va a analizar en qué medida éste modelo de última generación es mejor en desempeño en relación a uno clásico.

Para la estrategía empírica se emplean datos del índice de precios domésticos y precios internacionales de determinados bienes primarios. En cuanto a los variables domésticas se utilizan las ochenta y ocho clases del IPC de Uruguay, extraídas del Instituto Nacional de Estadística (INE).  Con respecto a los precios internacionales de los bienes primarios, se utilizan precios de tres bienes primarios (carne, soja y petróleo) y el promedio simple del precio de los alimentos. Los mismos son tomados del Fondo Monetario Internacional. Para el precio de la carne se toma la serie correspondiente a la carne de Australia y Nueva Zelanda en centésimos de dólar por tonelada, para el petróleo se toma el promedio de tres precios spot expresados en dólares por barril, para la soja
se toma el precio futuro de la soja en Chicago expresado en dólares por tonelada
métrica  y para los alimentos se realiza un promedio simple de los precios de commodities alimenticios disponibles. 