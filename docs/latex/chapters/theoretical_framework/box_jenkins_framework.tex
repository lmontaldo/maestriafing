\chapter{Modelo de selección de Box-jenkins}
El presente estudio se basa en la estrategia de Box-jenkins
\section{Concepto de estacionariedad}
Se dice que un proceso estocástico $y_t$ con media y varianza finitas es estacionario en covarianza para todo $t$ y $t-s$ cuando

\begin{align}
\label{eq:2_7}
E(y_t)&=E(y_{t-s})=\mu\\
\label{eq:2_8}
var(y_t)&=var(y_{t-s})=\sigma^2_y\\
\label{eq:2_9}
cov(y_t,y_{t-s})&=cov(y_{t-j},y_{t-j-s})=\gamma_s
\end{align}
donde $\mu, \sigma^2_y, \gamma_s$ con constantes.
   
En~\eqref{eq:2_8}, cuando $s=0$ se va a tener que $\gamma_0$ es la varianza de $y_t$. Que una serie temporal sea estacionaria en covarianza implica que tanto la media como todas sus autocovarianzas\footnote{También se habla de proceso debilmente estacionario o estacionario de segundo orden.} no están afectadas por cambios en los orígenes temporales. En modelos multivariados, como es el caso en este estudio, el término autocovarianza refiere a la covarianza entre $y_t$ y sus propios rezagos mientras que covarianza cruzada refiere a la covarianza entre series temporales.

Para series estacionarias de segundo orden, se define la autocorrelación de $y_t$ y $y_{t-s}$ como

\begin{equation*}
\rho_s\equiv\gamma_s/\gamma_0
\end{equation*} 

Como $\gamma_s/$ y $\gamma_0$ son independientes del tiempo, entonces los coeficientes de correlación $\rho_s$ también van a ser independientes del tiempo. Si bien la autocorrelación entre $y_t$ y $y_{t-1}$ podría ser distinta a la autocorrelación entre $y_t$ y $y_{t-2}$, la autocorrelación entre $y_t$ y $y_{t-1}$ va a ser igual a la autocorrelación entre $y_t$ y $y_{t-s-1}$. Asimismo, $\rho_0=1$.

Las autocovarianzas y las autocorrelaciones son herramientas esenciales en la metodología de Box-Jenkins(1976) para identificar y estimar los modelos de series temporales. Dado que se está en un marco de ecuaciones en diferencias, se tienen que cumplir condiciones de estabilidad en el sistema y esto implica que las raíces características asociadas al sistema de interés caígan dentro del círculo unitario. De modo que para cada serie temporal estacionaria, individualmente considerada, su correlograma, que consiste en graficar $\rho_s$ contra $s$ debe converger a cero.  Además, cuando se quiera eliminar el efecto de los valores intermedios $y_{t-1}, ..., y_{t-s+1}$  entre  $y_t$ y $y_{t-s}$ se puede emplear la autocorrelación parcial dado que el correlograma incluye las correlaciones indirectas del proceso autorregresivo. Las autocorrelaciones parciales se calculan como

\begin{align}
\label{eq: 2_35}
\phi_{11}&=\rho_1\\
\label{eq: 2_36}
\phi_{22}&=(\rho_2-\rho_1^2)/(1-\rho_1^2)\\
\nonumber
\text{para rezagos adicionales,}\\
\label{eq: 2_37}
\phi_{ss}&=\frac{\rho_s-\sum_{j=1}^{s-1}\phi_{s-1,j}\rho_{s-j}}{a-\sum_{j=1}^{s-1}\phi_{s-1,j}\rho_j}
\end{align}
donde $\phi_{sj}=\phi_{s-1,j}-\phi_{ss}\phi_{s-1,s-j}$ con $j=1,2,3,...,s-1$.

En la etapa de identificación, se analizan los gráficos de cada una de las series univariadas, sus correlogramas y las funciones de correlación parcial. Esto permite detectar datos atípicos, faltantes y cambios estructurales en las mismas. Las series no estacionarias, podrían presentar una evidente tendencia, medias y varianzas que no son constantes en el tiempo.

Una concepto fundamental de este enfoque es el de parsimonia. Implica que al adicionar coeficientes al modelo se aumenta el ajuste (el valor de $R^2$) a costa de la reducción de los grados de libertad del modelo. Box y Jenkins señalan que los modelos parsimoniosos generan mejoren predicciones que aquellos modelos sobreparametrizados. En este contexto se sugiere emplear tanto los criterios AIC como SBC como medidas apropiadas del ajuste general del modelo.

Otro concepto clave en este enforque es el de la invertibilidad. Se dice que $y_t$ es invertible si puede ser representada por un proceso autorregresivo convergente o de primer order.

Un supuesto fundamental del enfoque Box-Jenkins es que la estructura del proceso generativo de los datos no presenta cambios. Para evaluar esto para cada serie temporal se puede emplear la prueba de Chow. Sin embargo, el test de Chow y sus variantes asumen que el cambio se manifiesta en los datos, es decir, es evidente. Entre varias técnicas que se desarrollaron en la literatura, se podría usar la propuesta por Brown, Durbin, Evans (1975) que consiste en hallar la suma acumulada de los errores de predicción para cada serie y evaluar si es distinta de cero:

\begin{equation*}
CUSUM_N=\sum_{i=n}^{N}e_i(1)/\sigma_e,\quad N=n,...,T-1    
\end{equation*}
donde $n$ es la fecha para el primer error de predicción y $T$ la última observación del conjunto de datos y $\sigma_e$ la desviación estándar estimada de los errores de predicción.



\section{Ecuaciones en diferencias}

Una de las metodologías empleadas en este estudio para series temporales se basan en la teoría de las ecuaciones en diferencias. De forma general, las ecuaciones en diferencias expresan  valores de las variables de interés como funciones de sus propios valores rezagados, del tiempo y de otras variables.
En este sentido, se van a realizar estimaciones de ecuaciones en diferencias que contienen componentes estocásticos. 
Además, en los modelos de series temporales, se parte de descomponer a las series en distintos componentes: tendencia, ciclo y componente irregular. El primero se vincula al comportamiento de largo plazo de cada serie, el segundo representa los movimientos períodico regulares de las mismas, y el último es estocástico. Tanto la tendencia como el ciclo son funciones del tiempo mientras que el componente irregular es una función de sus propios rezagos y de un componente estocástico $\varepsilon_t$. El objetivo principal en los estudios de series temporales consiste en estimar el componente irregular estocástico y realizar predicciones, entre otras tareas.

