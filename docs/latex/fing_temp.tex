

\documentclass[a4paper,12pt]{article} % Acepta tamaños de letra de 10pt, 11pt y 12pt.

% ===========================================================================
% ============================ PREÁMBULO ====================================
% ===========================================================================

% *** LANGUAGE PACKAGES ***
%
\usepackage[utf8]{inputenc} % Este paquete hace que se puedan compilar símbolos del idioma español, como la letra "ñ" o un acento gráfico.
\usepackage[spanish,es-nodecimaldot]{babel} % Escribe en español los diferentes títulos por defecto del documento i.e. Índice, Referencias, etc.
										    % La opción es-nodecimaldot es para que los separadores decimales sean puntos.
     
% *** GEOMETRY PACKAGES ***
%
\usepackage{geometry}
\geometry{
total={210mm,297mm},
left=25mm,
right=25mm,
top=30mm,
bottom=30mm,
} % Se puede editar el tamaño de los márgenes de cada lado de la hoja
 
\usepackage{fancyhdr} % Paquete para editar el formato de la página
\renewcommand{\headrulewidth}{0.2pt} % Lineas debajo del encabezado
\renewcommand{\footrulewidth}{0.2pt} % Lineas arriba de los pie de página
\pagestyle{fancy}                    % Estilo de página                      
\cfoot{}                             % Se quita numeración de página en el centro, que es por defecto.                        
\lhead{Tesis Laura Montaldo}             % Encabezado izquierdo    
%\lfoot{\footnotesize }        % Pie de página izquierdo
\rfoot{Pág. \thepage} 			     % Pie de página derecho

% *** CITATION PACKAGES ***
%
\usepackage{cite} % Para citar referencias \cite{keylist}
%
% *** GRAPHICS RELATED PACKAGES ***
%
\usepackage{graphicx}         %Para trabajar con imagenes
\usepackage{float}            % Para poder poner figuras dentro de minipages
\usepackage{wrapfig}          % Para poner figuras dentro del texto
\graphicspath{{./figures_latex/}}  % Se indica ubicación de carpeta con imágenes para no indicarlo en cada imagen
\usepackage{caption}          % Para controlar mejor la colocación de leyendas en figuras y tablas.
\usepackage[percent]{overpic} % Para escribir texto/ecuaciones sobre figuras.  

% *** NUMBERING RELATED PACKAGES ***
%

\usepackage{chngcntr}             % Cambio como se representa la numeración de las imágenes, ecuaciones y tablas
							      % Ahora se nombran por sección reiniciando el conteo en cada sección
\counterwithin{figure}{section}
\counterwithin{equation}{section}
\counterwithin{table}{section}

% *** MATH PACKAGES ***
%
\usepackage{amsmath} % Para trabajar con ecuaciones, matrices y mucho más.
\usepackage{empheq}  % Similar al paquete anterior con algunas ventajas

% *** TABLES PACKAGES ***
%
\usepackage{multirow} % Para trabajar con celdas de varias filas de espesor 
\usepackage{multicol} % Para trabajar con celdas de varias columnas de ancho

% *** SUBFIGURE PACKAGES ***
%
\usepackage{subfig}   % Permite trabajar con subfiguras

% *** URL AND HYPERLINK PACKAGES ***
%
\usepackage{hyperref} % Este paquete hace que el índice tenga hiperreferencias
\hypersetup{colorlinks, urlcolor=cyan, citecolor=green, linkcolor=blue} % Color de referencias
\usepackage{url}      % Para hacer hiperreferencias a páginas web


% *** BOOKMARK PACKAGES ***
\usepackage[nottoc,notlot,notlof]{tocbibind} % Agrega en índice a las referencias

% ===========================================================================================
% ================================== INICIO DEL DOCUMENTO ===================================
% ===========================================================================================

\begin{document}
%
\renewcommand{\contentsname}{Tabla de contenidos}      % Cambia nombre al "índice"
\renewcommand{\tablename}{\bfseries Tabla}             % Cambia nombre de tablas
\renewcommand{\figurename}{\bfseries Figura}           % Cambia nombre de figuras
\newcommand{\subfigureautorefname}{\figureautorefname} % Para que al referenciar una subfigura aparezca "Figura"

% === Título del documento estilo informes de Facultad de Ingeniería ===
\begin{titlepage}
	
	\begin{center}
	%\begin{figure*}[h]
	% Se colocan logos institucionales con sus nombres
	%\noindent 
	%\captionsetup[subfigure]{labelformat=empty,justification=raggedright,singlelinecheck=false}
%\subfloat[{\large Universidad de la República Facultad de Ingeniería}]{\parbox{7cm}{\includegraphics[width=0.8in]{fing}}}%
	%\captionsetup[subfigure]{labelformat=empty,justification=raggedleft,singlelinecheck=false}
	%\hfill
	%\subfloat[{\large Instituto de Estructuras y Transporte Prof. Julio Ricaldoni}]{\parbox{9cm}{\raggedleft\vspace{0.3cm} \includegraphics[width=1in]{iet}}}%
	%\end{figure*} 
	
	\textsc{\Large  tesis maestría}\\[1.5cm] % Título del informe/memoria
	
	\textsc{\LARGE Primer versión}\\[0.5cm] % Subtítulo del informe/memoria
	
	%\includegraphics[width=5in]{structure} \\[1cm] % Figura del trabajo -Se puede comentar y dejar el espacio en blanco si se quisiera.
	% Autores y docentes
	
	\vspace{-0.5cm}
	
	\noindent
	\begin{flushleft} \large
		\emph{Autora:}\\
		Nombre \textsc{Montaldo} - C.I.: 3.512.962-7 \\

		
		\vspace{0.5cm}
		
		\emph{Tutor:} \\
		Nombre \textsc{Marco Scavino}
	\end{flushleft}
	\vfill
	
	% Fecha
	{\large \today}
	
	\end{center}
\end{titlepage}

\thispagestyle{empty} % Para que no aparezca numero de pagina
\newpage              % Salta a nueva página
\tableofcontents      % Crea índice (o tabla de contenidos). Hay que compilar 2 veces para que se actualice.
\thispagestyle{empty} % Para que no aparezca numero de pagina en esta carilla.
\newpage              % Se usa \newpage al final para que no ajuste todo el índice al tamaño de la hoja

\clearpage

\pagenumbering{arabic} % Define el tipo de numeración de las páginas.
%%%%%%%%%%%%%%%%%% Primer seccion %%%%%%%%%%%%%%%%%%%%%%%%%%%%%%%%
\section{Resumen \LaTeX}
\subsection{Introducción}
	
	En este documento se explican la gran mayoría de las cosas simples y básicas para hacer en \LaTeX. Luego al crear su propio documento el usuario puede copiar el código de este template y modificar los datos para usarlo con otro fin. 
	
	Este template es de libre distribución y se puede modificar según lo que uno necesite. En lo que sigue se va a explicar como escribir dentro de \LaTeX, por lo que luego de compilar uno puede volver al código y entender como se escribió el código. Vale aclarar que hay varias formas para lograr lo mismo, por lo que posiblemente hayan mejores o peores formas de escribir ciertas expresiones. Ésto aplica para todo en \LaTeX, desde como expresar una fórmula matemática, a colocar figuras o presentar una tabla.

\subsection{Editando texto}
	Hay diferentes tipos de \textit{Font Styles}, puedo tener palabras en \textbf{negrita} o en \textit{itálica} como también \underline{subrayadas}, se puede poner \emph{énfasis} o escribir \textsl{inclinado}. Está el estilo \texttt{máquina de escribir}, el de \textsc{pequeña capitalización} o sino \textsf{Sans Serif}.
	
	Respecto al tamaño de texto de alguna palabra, frase o párrafo, puedo escribir {\tiny super chiquito}, no tan {\scriptsize chiquito}, un poquito menos {\footnotesize chico}, un poco menos {\small chico}. Sino puedo escribir un poco más {\large grande}, otro poco más {\Large grande}, bastante {\LARGE grandioso}, sino {\huge gigante } y hasta {\Huge gigantesco}. Lo bueno de esto es que los tamaños de letra están relacionados por adjetivos respecto al tamaño de letra determinado al definir el tipo de clase de documento, por lo que si se llegara a cambiar, por ejemplo, de 12pt a 11pt, el tamaño de todos los textos del documento se ajustaría proporcionalmente, manteniendo la misma armonía de antes.
	
	Al trabajar con texto, para indicar que quiero hacer un punto y aparte, simplemente hay que escribir una línea y luego dejar otra en blanco, de esa forma \LaTeX ~entiende que tiene que hacer un punto y aparte. 


	Al hacer el punto y aparte de esta forma, la sangría se agrega automáticamente. Si se quiere indicar un salto de línea se puede escribir \verb|\\| y se sigue en la otra y no va a tener sangría la línea siguiente. Para forzar sangría se pone \verb|\indent| y para forzar que no haya \verb|\noindent|. Luego de una imagen o tabla se agrega sangría automáticamente. Otra forma de generar espacio vertical es con el comando \verb|\vspace{length}| donde \verb|length| puede por ejemplo ser 1.32cm.
	
	\vspace{1.32cm}
	
	Se coloca ahora un salto de página con el comando \verb|\newpage|.
	
	\newpage
	
\bibliographystyle{plain}

\begin{thebibliography}{5} %
\thispagestyle{empty}
%
\bibitem{Lenci}
Lenci, S. y Clementi, F. \emph{Simple Mechanical Model of Curved Beams by a 3D Approach.} \relax Journal of Engineering Mechanics, ASCE, 2009.
%
\bibitem{Onate}
Oñate, E. \emph{Structural Analysis with The Finite Element Method-Linear Statics,
Vol. 2: Plates and Shells} 1st~ed. \relax Springer-CIMNE, 2013.
%
\bibitem{Zienkiewicz}
Zienkiewicz, O.C. y Taylor, R.L. \emph{El método de los elementos finitos. Vol. 2: Mecánica de sólidos y fluidos. Dinámica y no linealidad.} 6ta~ed. \relax Elsevier, 2005.
%
\end{thebibliography}
\end{document}