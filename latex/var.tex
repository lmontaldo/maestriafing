\documentclass{article}

\begin{document}
%%%%%%%%%%%%%%%%%%%%%%%%%%%%%%%%%%%%%%%%%%%%%%%%%%%%%%%%%%%%%%%
%%%%%%%%%%%%%%%%%%%%%%%%%%%%%%%%%%%%%%%%%%%%%%%%%%%%%%%%%%%%%%%%%%%
%%%%%%%%%%%%%%%%%%%%%%%%%%%%%%%%%%%%%%%%%%%%%%%%%%%%%%%%%%%
\section{Modelos VAR: breve resumen}
Siguiendo a Enders(año), cuando no se tiene la certeza de si una variabla es exógena en un sistema, 
una solución en el sentido de funciones de transferencia es considerar a cada variable de forma simétrica.
Para el caso bivariado, se tiene el siguiente sistema:

\begin{eqnarray}
    y_{t}=b_{10}-b_{12}z_t+\gamma _{11}y_{t-1}+\gamma _{12}z_{t-1}+\varepsilon _{yt} \\
    z_{t}=b_{20}-b_{21}y_{t}+\gamma _{21}y_{t-1}+\gamma _{22}z_{t-1}+\varepsilon _{zt}
\end{eqnarray}
Otra forma: 


%%%%%%%%%%%%%%%%%%%%%%%%%%%%%%%%%%%%%%%
\section{Aplicaciones VAR}

%%%%%%%%%%%%%%%%%%%%
\end{document}
